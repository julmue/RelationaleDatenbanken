\documentclass{scrbook}

\usepackage{common}

% -----------------------------------------------------------------------------
% -----------------------------------------------------------------------------
% -----------------------------------------------------------------------------
\begin{document}

\tableofcontents

\newpage
% %%%%%%%%%%%%%%%%%%%%%%%%%%%%%%%%%%%%%%%%%%%%%%%%%%%%%%%%%%%%%%%%%%%%%%%%%%%%%
% Relationale Datenbanken -----------------------------------------------------
% %%%%%%%%%%%%%%%%%%%%%%%%%%%%%%%%%%%%%%%%%%%%%%%%%%%%%%%%%%%%%%%%%%%%%%%%%%%%%
\chapter{Relationale Datenbanken}


% %%%%%%%%%%%%%%%%%%%%%%%%%%%%%%%%%%%%%%%%%%%%%%%%%%%%%%%%%%%%%%%%%%%%%%%%%%%%%
% Relationale Algebra ---------------------------------------------------------
% %%%%%%%%%%%%%%%%%%%%%%%%%%%%%%%%%%%%%%%%%%%%%%%%%%%%%%%%%%%%%%%%%%%%%%%%%%%%%
\chapter{Relationale Algebra}

\section{Algebraische Struktur}
% Wiki, Algebraische Struktur
Eine \emph{algebraische Struktur} (auch \emph{Algebra}) ist eine Menge 
versehen mit Verknüpfungen (Operationen) auf dieser Menge.

\begin{definition}[Algebraische Struktur / Algebra]
Eine algebraische Struktur oder allgemeine Algebra ist ein geordnetes Paar
\begin{displaymath}
	(A,(f_i)_{i ∈ I})
\end{displaymath}
bestehend aus 
\begin{itemize}
\item $A$ (nichtleere) Grundmenge / Trägermenge
\item $(f_i)_{i ∈ I}$ Familie von inneren (endlichstelligen) Verknüpfungen auf $A$,
	auch Grundoperationen oder fundamentale Operationen genannt.
\end{itemize}
\end{definition}

\begin{definition}[Innere (endlichestellige) Verknüpfung]
Eine innere n-stellige Verknüpfung ist eine Funktion $f: A^n → A$,
die $n$ Elemente $a_1, \dots, a_n$ aus $A$ immer auf ein (eindutig bestimmtes
Element $b$ aus $A$ abbildet, $b$ ist dann das Bild von $(a_1, \dots, a_n)$ 
($b = f(a_1, \dots, a_n)$.
\end{definition}

\noindent
Sonderfälle:
\begin{itemize}
\item nullstellige innere Verknüpfung: Konstante (meist mit speziellem Konstantensymbol)
\item einstellige innere Verknüpfung: unäre Funktion von $A$ nach $A$
\item zweistellige innere Verknüpfung: Operation, binäre Funktion von $A$ nach $A$
\end{itemize}

\noindent
Meistens hat eine Algebra nur endlich viele fundamentale Operationen,
dann schreibt man für die Algebra einfach nur
\begin{displaymath} 
(A, f_1, \dots, f_n)
\end{displaymath}


% -----------------------------------------------------------------------------
% Wiki Relationale Algebra
\section{Einleitung Relationale Algebra}

In der Theorie der Datenbanken versteht man unter einer
\emph{Relationenalgebra} oder einer \emph{relationalen Algebra} eine formale Sprache,
mit der sich Abfragen über einem relationalen Schema formulieren lassen.
Die Operationen der Algebra erlauben es, Relationen miteinander zu verknüpfen 
oder zu reduzieren und komplexere Informationen daraus herzuleiten.

Die relationale Algebra definiert Operationen, die sich auf einer Menge von Relationen
andwenden lassen.
Damit lassen sich beispielsweise Relationen verknüpfen, filtern oder umbenennen.
Die Ergebnisse aller Operationen sind ebenfalls Relationen.
Aus diesem Grund bezeichnet man die Relationenalgebra als 
abgeschlossen. % ist nicht jede Algebra abgeschlossen?

Ihre Bedeutung hat die Relationenalgebra als theoretische Grundlage für 
Abfragesprachen in relationalen Datenbanken.
Hier werden die Operationen der relationalen Algebra in so genannten 
Datenbankoperatoren implementiert.

% Mehrere Basen
Für die relationale Algebra gibt es mehrere minimale Mengen von Operationen,
aus denen alle weiteren Operationen zusammengesetzt werden können.
Ein übliche minimale Basis besteht aus sechs Operationen:
\begin{enumerate}
\item Projektion
\item Selektion
\item Kreuzprodukt
\item Vereinigung
\item Differenz
\item Umbenennung
\end{enumerate}

% (Offene) Rekursion ist in der relationalen Algebra nicht möglich
Bestimmte Abfragesprachen erweitern die Mächtigkeit der relationalen Algebra:
beispielsweise ist in der relationalen Algebra die Bildung der transitiven Hülle
einer Relation (Vorgänger-Von) nicht gegeben.

Im Gegensatz zu Kalkülen ist die relationale Algebra sicher, d.h.
sie liefert in endlicher Zeit ein endliches Resultat.


% -----------------------------------------------------------------------------
\section{Operationen}

\subsection{Mengenoperationen}

Um Mengenoperationen auf der Relationen $R$ und $S$ durchführen zu können,
müssen beide miteinander kompatibel sein.
Die Typkompatibilität zweier Relationen ist gegeben, wenn
\begin{itemize}
\item $R$ und $S$ den gleichen Grad (Attributelementzahl) haben
\item der Wertebereich der Attribute von $R$ und $S$ identisch ist.
\end{itemize}

Die Typkompatibilität wird auch \emph{Vereinigungsverträglichkeit} genannt.

\paragraph{Vereinigung}

Bei der Vereinigung $R ⋃ S$ werden alle Typel der Relation $R$ 
mit allen Tupeln der Relation $S$ zu einer einzigen Relation vereint.
Voraussetzung dafür ist, dass $R$ und $S$ das gleiche Relationschema haben,
dass heißt, sie haben gleiche Attribute und Attributtypen.

\begin{definition}[Vereinigung]
\begin{displaymath}
R ⋃ S = \{t | t ∈ R ∨ t ∈ S \}
\end{displaymath}
\end{definition}

Duplikate werden gelöscht.

% \begin{example}
% 
% R:
% KundenNr 	| Vorname 	| Nachname 
% 23451		| Benji		| Mussler
% 63234		| Heike		| Graul
% 03526		| Angela	| Schneider
% 
% S: 
% KundenNr	| Vorname	| Nachname
% 62672		| Gert		| Nurmig
% 03526		| Angela	| Schneider
% 
% R ⋃ S:
% KundenNr 	| Vorname 	| Nachname 
% 23451		| Benji		| Mussler
% 63234		| Heike		| Graul
% 62672		| Gert		| Nurmig
% 03526		| Angela	| Schneider
%
% \end{example}


\paragraph{Schnittmenge / Intersection}

Das Ergebnis der Durchschnittsoperation $R ⋂ S$ sind alle Tupel,
die sich sowohl in $R$ als auch in $S$ finden lassen.
Voraussetzung für den Schnitt ist die Vereinigungsverträglichkeit von $R$ und $S$.

\begin{definition}[Schnittmenge]
\begin{displaymath}
R ⋂ S = \{ t | t ∈ R ∧ t ∈ S \}
\end{displaymath}
\end{definition}

% \begin{example}
% 
% R:
% KundenNr 	| Vorname 	| Nachname 
% 23451		| Benji		| Mussler
% 63234		| Heike		| Graul
% 03526		| Angela	| Schneider
% 
% S: 
% KundenNr	| Vorname	| Nachname
% 62672		| Gert		| Nurmig
% 03526		| Angela	| Schneider
% 
% R ⋂ S:
% KundenNr 	| Vorname 	| Nachname 
% 03526		| Angela	| Schneider
%
% \end{example}


\paragraph{Differenz}

Bei der Operation $R \\ S$ (auch $R - S$ werden aus der ersten Relation $R$
alle Tupel entfernt, die auch in der zweiten Relation $S$ vorhanden sind.
Voraussetzung ist die Vereinigungsverträglichkeit von $R$ und $S$.

Die Differenz ist keine monotone Operation, daher ist auch die relationale Algebra
im Vergleich zu anderen Anfragesprachen nicht monoton. (???)

\begin{definition}[Differenz]
\begin{displaymath}
R-S = R\\ S = \{ t | t ∈ R ∧ t ∉ S \}
\end{displaymath}
\end{definition}

% \begin{example}
% 
% R:
% KundenNr 	| Vorname 	| Nachname 
% 23451		| Benji		| Mussler
% 63234		| Heike		| Graul
% 03526		| Angela	| Schneider
% 
% S: 
% KundenNr	| Vorname	| Nachname
% 62672		| Gert		| Nurmig
% 03526		| Angela	| Schneider
% 
% R ⋂ S:
% KundenNr 	| Vorname 	| Nachname 
% 23451		| Benji		| Mussler
% 63234		| Heike		| Graul
%
% \end{example}



\subsection{Kartesisches Produkt (Kreuzprodukt)}

Das kartesische Produkt $R × S$ ist eine Operation,
welche dem kartesischen Produkt aus der Mengenlehre ähnelt.

Das Resultat des kartesischen Produkts ist die Menge aller Kombinationen der 
Tupel aus $R$ und $S$, jede Zeile der einen Tabelle wird mit jeder Zeile 
der anderen Tabelle kombiniert.

Wenn alle Merkmale (Spalten) verschieden sind, so umfasst die Resultatstabelle
die Summe der Merkmale der Ausgangstabellen.
Gleichnamige Merkmale der zwei Tabellen werden durch Voranstellen 
des Tabellennamens referenziert. Die Anzahl der Typel (Zeilen) 
in der Resultatstabelle ist das Ergebnis der Mutliplikation
der Zeilenanzahlen der Ausgangstabellen.

\begin{definition}
Zwei beliebige Relationen $R$ und $S$ sind gegeben.
Das kartesische Produkt ist definiert durch:
\begin{displaymath}
R × S = \{(a_1, a_2, \dots, a_n, b_1, b_2, \dots, b_m) | (a_1, a_2, \dots, a_n) ∈ R ∧ (b_1, b_2, \dots, b_m) ∈ S \}
\end{displaymath}
\end{definition}


% \begin{example}
% 
% R:
% KundenNr 	| Vorname 	| Nachname 
% 23451		| Benji		| Mussler
% 63234		| Heike		| Graul
% 03526		| Angela	| Schneider
% 
% S: 
% Farbe 	| Größe
% Grün		| XL
% Blau		| L
% 
% R × S:
% KundenNr 	| Vorname 	| Nachname	| Farbe		| Größe 
% 23451		| Benji		| Mussler	| Grün		| XL
% 23451		| Benji		| Mussler	| Blau		| L
% 63234		| Heike		| Graul		| Grün		| XL
% 63234		| Heike		| Graul		| Blau		| L
% 03526		| Angela	| Schneider 	| Grün		| XL
% 03526		| Angela	| Schneider 	| Blau		| L
%
% \end{example}


\subsection{Projektion}

Die Projektion entspricht der Projektionsabbildung aus der Mengenlehre
und kann auch Attributsbeschränkung genannt werden.
Sie extrahiert einzelne Attribute aus der ursprünglichen Attributmenge
und ist somit als eine Art Selektion auf Spaltenebene zu verstehen,
die Projektion blendet Spalten aus.
Wenn $β$ die Attributsliste ist, schreibt man $π_β(R)$. 
$β$ heißt auch Projektionsliste.
Duplikate in der Ergenisrelation werden eliminiert.

\begin{definition}[Projektion]
Sei $R$ eine Relation über $\{A_1, \dots, A_k\}$ und $β ⊆ \{ A_1, \dots, A_k \}$
\begin{displaymath}
π_β(R) = \{ t_β | t ∈ R \}
\end{displaymath}
\end{definition}

Voraussetzung: die angegebenen Spalten müssen in $R$ enthalten sein.



% \begin{example}
% 
% R:
% KundenNr 	| Vorname 	| Nachname 
% 93923		| Angela	| Meier
% 23451		| Benji		| Mussler
% 63234		| Heike		| Graul
% 03526		| Angela	| Schneider
%
% π[Vorname, Nachname] R:
% Vorname 	| Nachname 
% Angela	| Meier
% Benji		| Mussler
% Heike		| Graul
% Angela	| Schneider
%
% π[Vorname] R:
% Vorname 	
% Benji		
% Heike		
% Angela	


\subsection{Selektion / Restriktion}
Bei der Selektion kann man mit einem Vergleisausdruck (Prädikat) festlegen,
welche Tupel in die Ergebnismenge aufgenommen werden sollen.
Es werden also Tupel (Zeilen) ausgeblendet.
Man schreibt $σ_{Ausdruck}(R)$.
\emph{Ausdruck} heißt dann \emph{Selektionsbedingung}.

\begin{definition}[Selektion]
Sei $R$ eine Relation.
\begin{displaymath}
σ_{Ausdruck}(R) = \{ t | t ∈ R ∧ t erfüllt Ausdruck \} 
\end{displaymath}

\emph{Ausdruck} bezeichnet dabei eine \emph{Formel}. Diese kann bestehen aus:
\begin{itemize}
\item Konstantenselektionen \emph{Attribut $Ω$ Konstante, wobei $Ω$ ein passender Vergleichsoperator ist}
\item Attrubutselektions \emph{Attribut $Ω$ Attribut}
\item Eine Verknüpfung einer Vormel mit den logischen Prädikaten $∧,∨,¬$ (Klammerung wie üblich).
\end{itemize}
\end{definition}

Voraussetzung ist, dass jede angegebene Spalte über den Bedinungsoperator 
mit dem Vergleichswert vergleichbar sein.

% example
% R
% A 	B	C
% 1	2	4
% 4	6	7
% 1 	6	7
% 8	6	1
% 
% R[A=1]
% A 	B	C
% 1	2	4
% 1 	6	7
% 
% T[C>6]
% A 	B	C
% 4	6	7
% 1 	6	7



\paragraph{Join}

Ein Join bezeichnet die beiden hintereinander ausgeführten Operationen
kartesisches Produkt und Selektion.
Die Selektionsbedingung ist dabei üblicherweise ein Vergleich von Attributen
$A Ω B$, wobei $Ω$ ein passender Vergleisoperator ist.
Man bezeichnet den allgemeinen Vergund auch als $Ω$-Verbund (Theta-Verbund).
Ein Spezialfall des allgemeinen Verbundes ist der Equi-Join.

\begin{definition}
Für zwei Relationen $R(A_1, \dots, A_n)$ und $S(B_1, \dots, B_M)$ ist 
das Ergebnis das allgemeinen Verbundes mit einer Formel \emph{Ausdruck}
als Selektionsbedingung.
\begin{displaymath}
R ⨝_{Ausdruck} S = \{ r ⋃ s | r ∈ R ∧ s ∈ S ∧ Ausdruck \} = σ_{Ausdruck}(R × S)
\end{displaymath}
\end{definition}


% R:
% A	B	C	D
% 1	2 	3	4
% 4	5	6	7
% 7	8	9	0
% 
% S:
% E	F	G
% 1	2	3
% 7	8	9
% 
% R × S:
% A	B	C	D	E	F	G
% 1	2 	3	4	1	2	3
% 1	2 	3	4	7	8	9
% 4	5	6	7	1	2	3
% 4	5	6	7	7	8	9
% 7	8	9	0	1	2	3
% 7	8	9	0	7	8	9
% 
% Join(R, R.A <> S.E, S)
% A	B	C	D	E	F	G
% 1	2 	3	4	7	8	9
% 4	5	6	7	1	2	3
% 4	5	6	7	7	8	9
% 7	8	9	0	1	2	3




\paragraph{Equijoin}

Beim Equi-Jin (auch Gleichverbund) wird als erstes das kartesische Produkt gebildet.
Dann erfolgt die Selektion mit der Bedingung, dass der Inhalt bestimmter Spalten
identisch sein muss.
Der Equi-Join ist ein allgemeiner Verbund mit einer Formel der Form $A=B$.

\begin{definition}
Für die Relationen $R,S$ und dazugehörige Attribute $A$ (ist Attribut von $R$)
und $B$ (ist Attribut von $S$) ist der Equi-Join
\begin{displaymath}
R ⨝_{A=B} S = \{ (r,s) | r ∈ R ∧ S ∧ r_[a] = s_[B] \}
\end{displaymath}
\end{definition}

% R:
% A	B	C	D
% 1	2 	3	4
% 4	5	6	7
% 7	8	9	0
% 
% S:
% E	F	G
% 1	2	3
% 7	8	9
% 
% R × S:
% A	B	C	D	E	F	G
% 1	2 	3	4	1	2	3
% 1	2 	3	4	7	8	9
% 4	5	6	7	1	2	3
% 4	5	6	7	7	8	9
% 7	8	9	0	1	2	3
% 7	8	9	0	7	8	9
%
% R ⨝_{R.A=S.E} S:
% A	B	C	D	E	F	G
% 1	2 	3	4	1	2	3
% 7	8	9	0	7	8	9


\paragraph{Natural Join}
Der Natural Join setzt sich zusammen aus dem Equi-Join und einer zusätzlichen Ausblendung
der duplizierten Spalten (Projektionen).
Der Join erfolgt über die Attribute (Spalten) die in beiden
Relationen die gleiche Bezeichnung haben.
Gibt es keine gemeinsamen Attribute, so ist das Ergebnis des natürlichen Verbundes
das kartesische Produkt.
Der natürliche Verbund ist kommutativ und assoziativ,
es gilt $R⨝S = S⨝R$ und $R⨝(S⨝T) = (R ⨝ S) ⨝ T$,
was eine Rolle bei der Optimierung von Anfragen spielt.
Die Anzahl der Attribute der Ergebnisrelation ist die Summe der 
Anzahlen der beiden Ausgangsrelationen abzüglich die Anzahl der Verbundattribute.

\begin{definition} 

Für zwei Relationen $R(A_1, \dots, A_n, B_1, \dots, B_n)$
und $S(B_1, \dots, BN, C_1, \dots, C_n)$ ist das Ergebnis des natürlichen Verbundes
\begin{displaymath}
R⨝S = \{ r ⋃ s[C_1,\dots,C_n] | r ∈ R ∧ s ∈ S ∧ r[B_1,\dots,B_n] = s[B_1,\dots,B_n] \}
\end{displaymath}

\end{definition}

% R:
% A	B	C	D
% 1	2 	3	4
% 4	5	6	7
% 7	8	9	0
% 
% S:
% A	F	G
% 1	2	3
% 7	8	9
%
% Kartesisches Produkt R × S:
% A	B	C	D	A	E	F
% 1	2 	3	4	1	2	3
% 1	2 	3	4	7	8	9
% 4	5	6	7	1	2	3
% 4	5	6	7	7	8	9
% 7	8	9	0	1	2	3
% 7	8	9	0	7	8	9
%
% Equijoin R ⨝_{R.A=S.E} S:
% A	B	C	D	A	E	F
% 1	2 	3	4	1	2	3
% 7	8	9	0	7	8	9
%
% Natural Join (R,S)
% A	B	C	D	E	F
% 1	2 	3	4 	2	3
% 7	8	9	0	8	9


\paragraph{Semi Join}

Der Semi Join berechnet den Anteil eines Natural Joins, welcher nach einer
Reduktion auf die linke Relation übrig bleibt.

\begin{definition}
Für zwei Relationen $R(A_1, \dots, A_n,B_1, \dots, B_n)$ und 
$S(B_1, \dots, B_n,C_1, \dots, C_n)$ ist das Ergebnis des halben 
natürlichen Verbundes
\begin{displaymath}
R ⋉ S = \{ r | r ∈ R ∧ s ∈ S ∧ r[B_1, \dots, B_n] = s[B_1, \dots, B_n] \}
\end{displaymath}
\end{definition}

% R:
% A	B	C	D
% 1	2 	3	4
% 4	5	6	7
% 7	8	9	0
% 
% S:
% A	F	G
% 1	2	3
% 7	8	9
%
% Kartesisches Produkt R × S:
% A	B	C	D	A	E	F
% 1	2 	3	4	1	2	3
% 1	2 	3	4	7	8	9
% 4	5	6	7	1	2	3
% 4	5	6	7	7	8	9
% 7	8	9	0	1	2	3
% 7	8	9	0	7	8	9
%
% Equijoin R ⨝_{R.A=S.E} S:
% A	B	C	D	A	E	F
% 1	2 	3	4	1	2	3
% 7	8	9	0	7	8	9
%
% Semi Join (R,S) 
% A	B	C	D		
% 1	2 	3	4 		
% 7	8	9	0		
%


\paragraph{Outer Join}

Im Gegensatz zum Equi-Join werden beim Outer-Join auch die Tupel der linken
(left outer join) bzw. der rechten (right outer join) Tabelle in die
Ergebnisrelation mit aufgenommen, die keinen Join-Partner finden.  Die nicht
vorhandenen Attrubute der Join-Relation werden mit Nullwerten aufgefüllt.  Die
Kombination aus Left- und Right-Outer-Join wird Outer-Join oder Full-Outer-Join
genannt.
Dabei werden alle Tupel in die Ergebnisrelation aufgenommen und jene Attribute
eines Tupels mit Nullwerten aufgefüllt, die keinen Join Partner
in der jeweils anderen Relation gefungen haben.

Der Outer Join kann mit oder ohne Join-Bedingung verwendet werden (???).


% R:
% A	B	C	D
% 1	2 	3	4
% 4	5	6	7
% 7	8	9	0
% 
% S:
% A	F	G
% 1	2	3
% 7	8	9
%
% Kartesisches Produkt R × S:
% A	B	C	D	A	E	F
% 1	2 	3	4	1	2	3
% 1	2 	3	4	7	8	9
% 4	5	6	7	1	2	3
% 4	5	6	7	7	8	9
% 7	8	9	0	1	2	3
% 7	8	9	0	7	8	9
%
% Left Outer Join (R, R.A = S.A, S)

% Kartesisches Produkt R × S:
% A	B	C	D	E	F
% 1	2 	3	4	2	3
% 4	5	6	7	NULL	NULL
% 7	8	9	0	8	9



\paragraph{Umbenennung}

Durch diese Operation können Attrugyte und Relationen umbenannt werden.
Diese Operation ist wichtig, um
\begin{itemize}
\item Joins von unterschiedlichen benannten Relationen zu ermöglichen
\item kartesische Produkte zu ermöglichen, wo es gleiche Attributnamen gibt,
	insbesondere auch mit der gleichen Relation
\item Mengenoperationen zwischen Relationen mit unterschiedlichen Attributen zu ermöglichen.
\end{itemize}

Die Schreibweise ist $ρ_[alt → neu](R)$.

\begin{definition}
\begin{displaymath}
ρ_[alt → neu](R) = \{ t' | t'(R - alt) = t(R-alt) ∧ t'(neu)=t(alt) \}
\end{displaymath}
\end{definition}

Beispiel

% R:
% A	B 	C
% 1	2	3
% 4	5	6
% 
% R[B→X]:
% A	X	C
% 1	2	3
% 4	5	6



% \paragraph{Division}



\section{Erweiterung der relationalen Algebra}

Um andere Abfragesprachen, speziell SQL, vollständig in die relationale Algebra
abbilden zu können, ist die relationale Algebra nicht mächtig genug.
Es gibt z.B. keine Möglichkeit, die SQL-Operatoren 
\lstinline{GROUP-BY/HAVING}, Aggregatfunktionen und Nullwerte in die relationale
Algebra zu übersetzen.
Erweiterungen der relationalen Algebra ermöglichen eine vollständige Abbildung.






% %%%%%%%%%%%%%%%%%%%%%%%%%%%%%%%%%%%%%%%%%%%%%%%%%%%%%%%%%%%%%%%%%%%%%%%%%%%%%
% NORMALISIERUNG --------------------------------------------------------------
% %%%%%%%%%%%%%%%%%%%%%%%%%%%%%%%%%%%%%%%%%%%%%%%%%%%%%%%%%%%%%%%%%%%%%%%%%%%%%
\chapter{Normalisierung}


% %%%%%%%%%%%%%%%%%%%%%%%%%%%%%%%%%%%%%%%%%%%%%%%%%%%%%%%%%%%%%%%%%%%%%%%%%%%%%
% Darstellungsformen und Schemata ---------------------------------------------
% %%%%%%%%%%%%%%%%%%%%%%%%%%%%%%%%%%%%%%%%%%%%%%%%%%%%%%%%%%%%%%%%%%%%%%%%%%%%%
\chapter{Darstellungsformen und Schemata}


% %%%%%%%%%%%%%%%%%%%%%%%%%%%%%%%%%%%%%%%%%%%%%%%%%%%%%%%%%%%%%%%%%%%%%%%%%%%%%
% SQP -------------------------------------------------------------------------
% %%%%%%%%%%%%%%%%%%%%%%%%%%%%%%%%%%%%%%%%%%%%%%%%%%%%%%%%%%%%%%%%%%%%%%%%%%%%%
\chapter{SQL}


% %%%%%%%%%%%%%%%%%%%%%%%%%%%%%%%%%%%%%%%%%%%%%%%%%%%%%%%%%%%%%%%%%%%%%%%%%%%%%
% Entwurfsmethodik ------------------------------------------------------------
% %%%%%%%%%%%%%%%%%%%%%%%%%%%%%%%%%%%%%%%%%%%%%%%%%%%%%%%%%%%%%%%%%%%%%%%%%%%%%
\chapter{Entwurfsmethodik}

\end{document}
